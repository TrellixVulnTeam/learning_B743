% https://www.overleaf.com/learn/latex/Theorems_and_proofs#Introduction
\documentclass{article}
\usepackage[utf8]{inputenc}

\usepackage{amsmath}
\usepackage{amssymb}
\usepackage{amsthm}

\newtheorem{theorem}{Theorem}[section]%reset counter every new section
\newtheorem{corollary}{Corollary}[theorem]%reset counter every new theorem
\newtheorem{lemma}{Lemma}%reuse theorem counter
\numberwithin{lemma}{section}

% \theoremstyle{definition}
\theoremstyle{remark}
\newtheorem*{remark}{Remark}

\begin{document}
\section{Introduction}
Theorems can easily be defined

\begin{theorem}
Let $f$ be a function whose derivative exists in every point, then $f$ is a continuous function.
\end{theorem}

\begin{theorem}[Pythagorean theorem]
\label{pythagorean}
This is a theorema about right triangles and can be summarised in the next equation
\[ x^2 + y^2 = z^2 \]
\end{theorem}

And a consequence of theorem \ref{pythagorean} is the statement in the next corollary.

\begin{corollary}
There's no right rectangle whose sides measure 3cm, 4cm, and 6cm.
\end{corollary}

You can reference theorems such as \ref{pythagorean} when a label is assigned.

\begin{remark}
    This statement is true, I guess.
\end{remark}

\setcounter{lemma}{2}
\begin{lemma}
Given two line segments whose lengths are $a$ and $b$ respectively there is a real number $r$ such that $b=ra$.
\end{lemma}

\begin{proof}
    To prove it by contradiction try and assume that the statemenet is false,
    proceed from there and at some point you will arrive to a contradiction.
\end{proof}

\end{document}
